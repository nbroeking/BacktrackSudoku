% This is "sig-alternate.tex" V1.9 April 2009
% This file should be compiled with V2.4 of "sig-alternate.cls" April 2009
%
% This example file demonstrates the use of the 'sig-alternate.cls'
% V2.4 LaTeX2e document class file. It is for those submitting
% articles to ACM Conference Proceedings WHO DO NOT WISH TO
% STRICTLY ADHERE TO THE SIGS (PUBS-BOARD-ENDORSED) STYLE.
% The 'sig-alternate.cls' file will produce a similar-looking,
% albeit, 'tighter' paper resulting in, invariably, fewer pages.
%
% ----------------------------------------------------------------------------------------------------------------
% This .tex file (and associated .cls V2.4) produces:
%       1) The Permission Statement
%       2) The Conference (location) Info information
%       3) The Copyright Line with ACM data
%       4) NO page numbers
%
% as against the acm_proc_article-sp.cls file which
% DOES NOT produce 1) thru' 3) above.
%
% Using 'sig-alternate.cls' you have control, however, from within
% the source .tex file, over both the CopyrightYear
% (defaulted to 200X) and the ACM Copyright Data
% (defaulted to X-XXXXX-XX-X/XX/XX).
% e.g.
% \CopyrightYear{2007} will cause 2007 to appear in the copyright line.
% \crdata{0-12345-67-8/90/12} will cause 0-12345-67-8/90/12 to appear in the copyright line.
%
% ---------------------------------------------------------------------------------------------------------------
% This .tex source is an example which *does* use
% the .bib file (from which the .bbl file % is produced).
% REMEMBER HOWEVER: After having produced the .bbl file,
% and prior to final submission, you *NEED* to 'insert'
% your .bbl file into your source .tex file so as to provide
% ONE 'self-contained' source file.
%
% ================= IF YOU HAVE QUESTIONS =======================
% Questions regarding the SIGS styles, SIGS policies and
% procedures, Conferences etc. should be sent to
% Adrienne Griscti (griscti@acm.org)
%
% Technical questions _only_ to
% Gerald Murray (murray@hq.acm.org)
% ===============================================================
%
% For tracking purposes - this is V1.9 - April 2009

\documentclass{sig-alternate}
  \pdfpagewidth=8.5truein
  \pdfpageheight=11truein
\usepackage{algorithm}
\usepackage[noend]{algpseudocode}
\usepackage{graphicx}

\begin{document}
%
% --- Author Metadata here ---
\conferenceinfo{SAC'15}{January 1 - April 31, 2016, Boulder, CO.}
\CopyrightYear{2016} % Allows default copyright year (2002) to be over-ridden - IF NEED BE.
\crdata{978-1-4503-3196-01/2016/01}  % Allows default copyright data (X-XXXXX-XX-X/XX/XX) to be over-ridden.
% --- End of Author Metadata ---

\title{Backtracking Sudoku Solver
\titlenote{(Produces the permission block, and
copyright information). For use with
SIG-ALTERNATE.CLS. Supported by ACM.}}
%
% You need the command \numberofauthors to handle the 'placement
% and alignment' of the authors beneath the title.
%
% For aesthetic reasons, we recommend 'three authors at a time'
% i.e. three 'name/affiliation blocks' be placed beneath the title.
%
% NOTE: You are NOT restricted in how many 'rows' of
% "name/affiliations" may appear. We just ask that you restrict
% the number of 'columns' to three.
%
% Because of the available 'opening page real-estate'
% we ask you to refrain from putting more than six authors
% (two rows with three columns) beneath the article title.
% More than six makes the first-page appear very cluttered indeed.
%
% Use the \alignauthor commands to handle the names
% and affiliations for an 'aesthetic maximum' of six authors.
% Add names, affiliations, addresses for
% the seventh etc. author(s) as the argument for the
% \additionalauthors command.
% These 'additional authors' will be output/set for you
% without further effort on your part as the last section in
% the body of your article BEFORE References or any Appendices.

\numberofauthors{1} %  in this sample file, there are a *total*
% of EIGHT authors. SIX appear on the 'first-page' (for formatting
% reasons) and the remaining two appear in the \additionalauthors section.
%
\author{
% You can go ahead and credit any number of authors here,
% e.g. one 'row of three' or two rows (consisting of one row of three
% and a second row of one, two or three).
%
% The command \alignauthor (no curly braces needed) should
% precede each author name, affiliation/snail-mail address and
% e-mail address. Additionally, tag each line of
% affiliation/address with \affaddr, and tag the
% e-mail address with \email.
%
% 1st. author
\alignauthor
Nicolas C. Broeking\\
       \affaddr{University of Colorado at Boulder}\\
       \affaddr{Advanced Algorithms}\\
       \affaddr{CSCI 5454}\\
       \email{nibr3402@colorado.edu}
% 2nd. author
}


\maketitle
\begin{abstract}
I implemented a backtracking sudoko solver. Not yet actually but I really hope I end up with one. 
\end{abstract}

% A category with the (minimum) three required fields
\category{H.4}{Algorithms}{Sudoku}{Backtracking}{Search}
%A category including the fourth, optional field follows...

\terms{Search}

\keywords{Sudoku, Search, Backtracking Search}

\section{Introduction}
Sudoku is a popular game played by puzzle enthusiasts all over the world where the player takes 
a grid of size $n^2 x n^2$ with some values filled in and then tries to fill in the rest of the grid. TODO: CITATION
I am able to apply a backtraking algorithm to solve the sudoku grid. 

\section{Backtrack Solver for Sudoku}

To play sudoku one first starts with a grid of size $n^2 x n^2$ where n is any integer $> 0$ however the most common game is 
played when $n = 3$ or a 9x9 grid. The game maker can fill in some values or constraints and then the player has to fill in
the remaining squares leaving the original constraints un touched. The player has won when the following conditions are met.

The premise of the game is simple find a solution where three rules are satisified.
\begin{itemize}
\item{All blocks are filled in with an integer x where $0 < x \le n^2$}
\item{There are no duplicate numbers in a given row}
\item{There are no duplicate numbers in a given col}
\item{There are no duplicate numbers in any given sub square}
\end{itemize}
A subsquare can be defined as a group of non overlapping blocks of size $n^2$

\subsection{Problem}
The sudoku solving problem is an NP-Complete problem TODO:CITATION. We can take
advantage of this fact to develop a backtracking algorithm to solve the problem. This solution
essentially boils down to a graph coloring where each block is a node and a block can be colored any number x where $0 < x \le n^2$. TODO:CITATION
As long as the coloring follows the rules of the game above. 

%---------------------------------------------------
\subsection{Algorithm}
The algorithm looks like such where G is the game matrix the box is the current node to color. The box number corisponds to the frame starting at the top left, box 0 and going to the bottom right box $n^2 - 1$ numbered horizontally.
\begin{algorithm}
\caption{Sudoku Backtracking}\label{solve}
\begin{aglorithmic}[1]
\Procedure{solve}{G}{box}

\If regect(G)
\State return None
\EndIf

\If accept(G)
\State return G
\EndIf

\If $box \ge n^4$
\State return None
\Endif

\State x, y = getIndecies(box, size)

\If !self.isempty(G[x][y])
\State return solve(G, box+1)
\EndIf

\State $G' = deepCopy(G)$

\For{ index $i < n^2$}
\State $G'$[x][y] = i
\State solution = solve($G'$, box + 1)
\If result !=None:
\State return Result
\EndIf
\EndFor

\State Return None
\EndProcedure
\end{algorithmic}
\end{algorithm}
TODO:CITATION

The above algorithm is dependent on the following sub routines

%Reject sub routine
The reject subroutine takes a Graph G and then checks to see if there is there is
a possible solution given the State. For example if the graph passed to it has two threes in a single column then reject will return True saying that there is no solution because you can't have more than one of each number in any given column. 

This subroutine is extremly powerful becuase it allows us to prune the tree for huge subtrees at a time. Given any two squares of the same value there are potentially $9^(n^4 - 2)$ sub states that can we don't need to search because we know they can't be valid. 

The reject subroutine does stuff and looks like.
\subsubsection{reject}
\begin{algorithm}
\caption{Reject}\label{reject}
\begin{algorithmic}[1]
\Procedure{reject}

\For{ index $i < n^2$}
\State $x = set()$
\For{ index $j < n^2$}
\If matrix[i][j] > 0
\If matrix[i][j] in x
\State return True
\EndIf
\State x.add(matrix[i][j])
\EndIf
\EndFor

\For{ index $j < n^2$}
\State $y = set()$
\For{ index $i < n^2$}
\If matrix[i][j] > 0
\If matrix[i][j] in y
\State return True
\EndIf
\State y.add(matrix[i][j])
\EndIf
\EndFor

\For{ index $offsetx < n^2$}
\For{ index $offsety < n^2$}
\State valid = set()
    \For{ index $i < n^2$}
    \For{ index $j < n^2$}
    \State indexX = offsetx*n + i
    \State indexY = offsetY*n + j

    \If $matrix[indexX][indexY] > 0$
    \If matrix[indexx][indexy] in valid 
    \State return True
    \EndIf
    \State valid.add(matrix[indexX][indexY]
    \EndIf
    
    \EndFor
    \EndFor
\EndFor
\EndFor
\State return False
\EndProcedure
\end{algorithmic}
\end{algorithm}
TODO:CITATION

%Accept sub routine
\subsubsection{accept}
The accept subroutine looks very similar to reject but instead of determining if there is a possible solution it confims that G is a valid solution. Accept takes only G as a parameter. For accept to return true all of the conditions for the win must be satisfied. The algorithm looks like the following.

\begin{algorithm}
\caption{Accept}\label{accept}
\begin{algorithmic}[1]
\Procedure{accept}

\For{ index $i < n^2$}
\State $x = set()$
\For{ index $j < n^2$}
\If matrix[i][j] > 0
\If matrix[i][j] in x
\State return False
\EndIf
\State x.add(matrix[i][j])
\Else
\State return False
\EndIf
\EndFor

\For{ index $j < n^2$}
\State $y = set()$
\For{ index $i < n^2$}
\If matrix[i][j] > 0
\If matrix[i][j] in y
\State return False
\EndIf
\State y.add(matrix[i][j])
\Else
\State return False
\EndIf
\EndFor

\For{ index $offsetx < n^2$}
\For{ index $offsety < n^2$}
\State valid = set()
    \For{ index $i < n^2$}
    \For{ index $j < n^2$}
    \State indexX = offsetx*n + i
    \State indexY = offsetY*n + j

    \If $matrix[indexX][indexY] > 0$
    \If matrix[indexx][indexy] in valid 
    \State return False
    \EndIf
    \State valid.add(matrix[indexX][indexY]
    \Else
    \State return False
    \EndIf
    
    \EndFor
    \EndFor
\EndFor
\EndFor
\State return True
\EndProcedure
\end{algorithmic}
\end{algorithm}
TODO:CITE

\subsubsection{getIndecies}

\subsubsection{isempty}

\subsubsection{deepCopy}

%---------------------------------------------------
\subsection{Correctness}

%---------------------------------------------------
\subsection{Runtime for size N}

%---------------------------------------------------
\subsection{Runtime for constraints}

%---------------------------------------------------
\subsection{Inputs} 

%---------------------------------------------------
\subsubsection{Inputs Worst Case}
\subsubsection{Inputs Average Case}
\subsubsection{Inputs Best Case}

 
\subsection{Citations}
Citations to articles \cite{bowman:reasoning,
clark:pct, braams:babel, herlihy:methodology},
conference proceedings \cite{clark:pct} or
books \cite{salas:calculus, Lamport:LaTeX} listed
in the Bibliography section of your
article will occur throughout the text of your article.
You should use BibTeX to automatically produce this bibliography;
you simply need to insert one of several citation commands with
a key of the item cited in the proper location in
the \texttt{.tex} file \cite{Lamport:LaTeX}.
The key is a short reference you invent to uniquely
identify each work; in this sample document, the key is
the first author's surname and a
word from the title.  This identifying key is included
with each item in the \texttt{.bib} file for your article.

The details of the construction of the \texttt{.bib} file
are beyond the scope of this sample document, but more
information can be found in the \textit{Author's Guide},
and exhaustive details in the \textit{\LaTeX\ User's
Guide}\cite{Lamport:LaTeX}.

This article shows only the plainest form
of the citation command, using \texttt{{\char'134}cite}.
This is what is stipulated in the SIGS style specifications.
No other citation format is endorsed or supported.

\section{Conclusions}
I concluded stuff
%\end{document}  % This is where a 'short' article might terminate

%ACKNOWLEDGMENTS are optional
\section{Acknowledgments}
This section is optional; it is a location for you
to acknowledge grants, funding, editing assistance and
what have you.  In the present case, for example, the
authors would like to thank Gerald Murray of ACM for
his help in codifying this \textit{Author's Guide}
and the \textbf{.cls} and \textbf{.tex} files that it describes.

%
% The following two commands are all you need in the
% initial runs of your .tex file to
% produce the bibliography for the citations in your paper.
\bibliographystyle{abbrv}
\bibliography{sigproc}  % sigproc.bib is the name of the Bibliography in this case
% You must have a proper ".bib" file
%  and remember to run:
% latex bibtex latex latex
% to resolve all references
%
% ACM needs 'a single self-contained file'!
%
%APPENDICES are optional
%\balancecolumns
\appendix
%Appendix A
\section{Headings in Appendices}
The rules about hierarchical headings discussed above for
the body of the article are different in the appendices.
In the \textbf{appendix} environment, the command
\textbf{section} is used to
indicate the start of each Appendix, with alphabetic order
designation (i.e. the first is A, the second B, etc.) and
a title (if you include one).  So, if you need
hierarchical structure
\textit{within} an Appendix, start with \textbf{subsection} as the
highest level. Here is an outline of the body of this
document in Appendix-appropriate form:
\subsection{Introduction}
\subsection{The Body of the Paper}
\subsubsection{Type Changes and  Special Characters}
\subsubsection{Math Equations}
\paragraph{Inline (In-text) Equations}
\paragraph{Display Equations}
\subsubsection{Citations}
\subsubsection{Tables}
\subsubsection{Figures}
\subsubsection{Theorem-like Constructs}
\subsubsection*{A Caveat for the \TeX\ Expert}
\subsection{Conclusions}
\subsection{Acknowledgments}
\subsection{References}
Generated by bibtex from your ~.bib file.  Run latex,
then bibtex, then latex twice (to resolve references)
to create the ~.bbl file.  Insert that ~.bbl file into
the .tex source file and comment out
the command \texttt{{\char'134}thebibliography}.
\end{document}
